%!LW recipe=xelatex
\documentclass[twoside]{article}
\usepackage{braket}
\usepackage{mathtools}
\usepackage{amssymb}
\usepackage{lmodern}
\usepackage{graphicx}
\usepackage[a4paper,margin=0.5in,includehead,]{geometry}
\usepackage{fancyhdr}
\usepackage[UTF8]{ctex}

\begin{document}
\title{量子光学03月12日第一次作业}
\author{Author}
\pagestyle{fancy}
\makeatletter
\fancyhead[L]{\@title}
\fancyhead[R]{\@author}
\makeatother
\setlength{\parindent}{0pt}

\section*{Answer for 2.3}

\begin{enumerate}
    \item[(a)] 与x轴成$+45^\circ$线偏振光
    \item[(b)] 与x轴成$+30^\circ$线偏振光
    \item[(c)] 左旋圆偏振光
    \item[(d)] 右旋圆偏振光
    \item[(e)] 长轴在x轴上的左旋椭圆偏振光, $a/b = \sqrt{3}$
    \item[(f)] 长轴在与x轴成$+45^\circ$的左旋椭圆偏振光, $a/b = \sqrt{2} + 1$
\end{enumerate}

\section*{Answer for 2.4}

有 $L = 5 \times 10^6 $km, $d = 30$cm, $ \lambda = 1064$nm

光经过衍射后光强大部分集中在直径为$d^\prime = 2 * \tan\theta_{min} L \approx 2 * \sin\theta_{min} L \approx 1.22 \frac{\lambda}{d} L$的圆内.
接收器接收到的功率$P$与光源的功率$P_{sou}$的比值满足接收器面积与衍射后光照范围的比值. 于是有:

$$
    P = P_{sou} \times \left(\frac{d^\prime}{d}\right)^2 = 4.8 \times 10^{-11} W
$$
\section*{Answer for 2.6}

\subsection*{2.6 (a)}


有多普勒展宽的关系式:
$$
    \Delta \omega = \frac{4\pi}{\lambda}\left[\frac{\left(2 \ln 2\right) k_B T}{m}\right]^{\frac{1}{2}}
$$
对于高斯线型, 有最大相干时间:
$$
    \tau = \frac{\left(8\pi\ln 2\right)^\frac{1}{2}}{\Delta \omega}
$$
所以最大光程差为其相干长度:
$$L_c = c \tau \approx 0.35 m$$

\subsection*{2.6 (b)}

同理有:

$$L_c = c \frac{1}{\Delta \omega} \approx 47.75 m $$

\section*{Answer for 2.10}

若要实现倍频, 需要相位匹配, 要求满足以下关系:
$$
    \frac{1}{\left(n^\omega_o\right)^2} = \frac{\sin^2 \theta}{\left(n^{2\omega}_e\right)^2} + \frac{\cos^2 \theta}{\left(n^{2\omega}_o\right)^2}
$$

代入计算可知 $\theta \approx 41^\circ$

\end{document}
