%!LW recipe=xelatex
\documentclass[twoside]{article}
\usepackage{braket}
\usepackage{mathtools}
\usepackage{amssymb}
\usepackage{lmodern}
\usepackage{graphicx}
\usepackage[a4paper,margin=0.5in,includehead,]{geometry}
\usepackage{fancyhdr}
\usepackage[UTF8]{ctex}
\usepackage{float}
% \usepackage{layouts}


\usepackage{pgfplots}
\begin{document}
\title{量子光学06月04日第七次作业}
\author{Author}
\pagestyle{fancy}
\makeatletter
\fancyhead[L]{\@title}
\fancyhead[R]{\@author}
\makeatother
\setlength{\parindent}{0pt}


% \providecommand{\mathdefault}[1]{#1}

\section*{Answer for 10.5}

\subsection*{(a)}

\begin{equation}
    Q \gg \left(\frac{4 c \epsilon \hbar \pi V_0}{\lambda \mu_{12}^2}\right)^{1/2} \Rightarrow Q_{min} = 5 \times 10^{7}
\end{equation}

\subsection*{(b)}

\begin{equation}
    \begin{rcases}
        \Delta \omega = \frac{\pi c}{n\mathcal{F} L_{cav}} \\
        Q = \frac{\omega}{\Delta \omega} \gg \left(\frac{4 c \epsilon \hbar \pi V_0}{\lambda \mu_{12}^2}\right)^{1/2}
    \end{rcases}
    \Rightarrow
    \mathcal{F} \gg \frac{\lambda}{2 n L_{cav}}\left(\frac{4 c \epsilon \hbar \pi V_0}{\lambda \mu_{12}^2}\right)^{1/2}
    \Rightarrow
    \mathcal{F}_{min} = 60000
\end{equation}

\subsection*{(c)}

\begin{equation}
    \begin{rcases}
        \tau_{cav} = \frac{nL_{cav}}{c\left(1-R\right)}            \\
        % \kappa = \frac{1}{\tau_{cav}}                              \\
        \Delta \omega = \left(\tau_{cav}\right)^{-1} \equiv \kappa \\
        Q = \frac{\omega}{\Delta \omega} \gg \left(\frac{4 c \epsilon \hbar \pi V_0}{\lambda \mu_{12}^2}\right)^{1/2}
        % \mathcal{F} = \frac{\pi \left(R_1 R_2\right)^{1/4} }{1- \sqrt{R_1 R_2}}                                             \\
        % \mathcal{F} \gg \frac{\lambda}{2 n L_{cav}}\left(\frac{4 c \epsilon \hbar \pi V_0}{\lambda \mu_{12}^2}\right)^{1/2} \\
        % R = R_1 = R_2
    \end{rcases}
    \Rightarrow
    R \gg  1 - \frac{2\pi n L_{cav}}{\lambda} \left(\frac{\lambda \mu_{12}^2}{4 c \epsilon \hbar \pi V_0}\right)^{1/2}
    \Rightarrow
    R_{min} = 99.995\%
\end{equation}

\section*{Answer for 10.8}

\begin{equation}
    F_p = \frac{3Q \left(\lambda/n\right)^3}{4\pi^2 V_0} = \frac{3\left(\lambda /n\right)^3 \left(\lambda^2 - \frac{1}{4}\Delta \lambda^2\right)}{4 \pi^2 V_0 \lambda \Delta \lambda } \approx 4.1
\end{equation}

\section*{Answer for 10.11}

\subsection*{(a)}
% \begin{equation}
%     \begin{rcases}
%         g_0 =6.3 \times 10^6 \text{rad} \cdot s^{-1} \\
%         \sqrt{N} g_0 \gg \kappa                      \\
%         \tau_{cav} = \frac{1}{\kappa}
%     \end{rcases}
%     \Rightarrow
%     \tau_{cav} \gg \frac{1}{\sqrt{N}g_0} \approx 70.48 \text{ns}
%     \Rightarrow
%     \tau_{cav} \sim 100 \text{ns}
% \end{equation}


\begin{equation}
    \begin{rcases}
        \Delta \omega = \frac{1}{\tau_{cav}} \\
        \Delta \omega = \frac{\pi c}{n \mathcal{F}L_{cav}}
    \end{rcases}
    \Rightarrow
    \tau_{cav} = \frac{n \mathcal{F} L_{cav}}{\pi c} \approx 88.28 \text{ns}
    \Rightarrow
    \tau_{cav} \sim 100 \text{ns}
\end{equation}


\subsection*{(b)}

\begin{equation}
    \Delta \Omega^{vac} = 2 \sqrt{N} g_0 > \kappa \Rightarrow N > \left(\frac{1}{2g_0 \tau}\right)^2 \approx 25
\end{equation}

\end{document}
