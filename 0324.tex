%!LW recipe=xelatex
\documentclass[twoside]{article}
\usepackage{braket}
\usepackage{mathtools}
\usepackage{amssymb}
\usepackage{lmodern}
\usepackage{graphicx}
\usepackage[a4paper,margin=0.5in,includehead,]{geometry}
\usepackage{fancyhdr}
\usepackage[UTF8]{ctex}

\begin{document}
\title{量子光学03月24日第二次作业}
\author{Author}
\pagestyle{fancy}
\makeatletter
\fancyhead[L]{\@title}
\fancyhead[R]{\@author}
\makeatother
\setlength{\parindent}{0pt}

\section*{Answer for 3.4}

\begin{equation*}
    \hat{l} =\hat{r} \times \hat{p} = \varepsilon_{ijk} \hat{i} r_j p_k  \Rightarrow \hat{l_i} = \varepsilon_{ijk} r_j p_k
\end{equation*}

\begin{equation*}
    \begin{split}
        \left[\hat{l_x}, \hat{l_y}\right] = & \left[r_y p_z - r_z p_y, r_z p_x - r_x p_z\right]                                                                               \\
        =                                   & \left[r_y p_z, r_z p_x\right] - \left[r_y p_z, r_x p_z\right] - \left[r_z p_y, r_z p_x\right] + \left[r_z p_y, r_x p_z\right]   \\
        =                                   & \left[r_y, r_z p_x\right] p_z  + r_y \left[p_z, r_z p_x\right] - \left[r_y, r_x p_z\right] p_z - r_y \left[ p_z, r_x p_z\right] \\
                                            & - \left[r_z, r_z p_x\right] p_y - r_z \left[p_y, r_z p_x\right] + \left[r_z, r_x p_z\right] p_y + r_z \left[p_y, r_x p_z\right] \\
        =                                   & 0 + r_y p_x (-i\hbar) - 0 - 0 - 0 - 0 + 0 + r_x p_y (i\hbar) + 0                                                                \\
        =                                   & i\hbar l_z
    \end{split}
\end{equation*}

同理有:
\begin{equation*}
    \left[\hat{l_y}, \hat{l_z} \right] = i\hbar \hat{l_x} \qquad \left[\hat{l_z}, \hat{l_x} \right] = i\hbar \hat{l_x}
\end{equation*}

\section*{Answer for 3.6}

对于激发态电子组态为(1s, 2p)的氦原子, 有
$$
    l_1 = 0, \quad l_2 = 1, \quad S_1 = S_2 = \frac{1}{2}
$$
所以:
$$
    L = 1, \quad S = 0\, \text{or} \, 1, \quad J = 0, 1, 2
$$
该组态所有可能的原子能级为: $0 P_{1}$, $^3 P_{0}$, $^3P_{1}$, $^3P_{2}$

\section*{Answer for 3.7}

\subsection*{(a)}

对超精细结构有:

$$\Delta E_{HFS} = A(J) \frac{\hbar^2}{2} \left[F(F+1) - J(J+1) - I(I+1)\right]$$

对于不同的$F$, 其能级差为
$$\Delta E_{HFS} - \Delta E_{HFS}^\prime =  A(J) \frac{\hbar^2}{2} \left[F(F+1) - F^\prime(F^\prime + 1)\right]$$

有$F = F^\prime + 1$, 所以:

$$\Delta E_{HFS} - \Delta E_{HFS}^\prime =  A(J) \frac{\hbar^2}{2} \left[F(F+1) - (F - 1)F\right] = A(J) \hbar^2 F $$

能级差正是正比于$F$

\subsection*{(b)}

自旋-轨道角动量相互作用导致的能量偏移也同样符合类似的关系:

$$\Delta E_{so} = C^\prime \left[J(J+1) - L(L+1) - S(S+1)\right] \Rightarrow \Delta E_{so J} - \Delta E_{so J-1} = 2 C^\prime J$$

\subsection*{(c)}

\subsubsection*{(i)}

对于纳的$3p ^2P_{3/2}$能级, 其电子角动量$J$为$\frac{3}{2}$, 包括核自旋的总角动量$\hat{F} = \hat{J} + \hat{I}$, 其中$\hat{I}$为核自旋角动量.
由于包括四条超精细能级, 所以该总角动量$\hat{F}$有四个值, 若$I = \frac{1}{2}$, 则$F$仅能为$1$和$2$, 仅有两条超精细能级, 不符合情况.
仅当$I \geqslant \frac{3}{2}$时, 才有可能有四条超精细能级.

\subsubsection*{(ii)}

有:
\begin{align*}
     & \Delta E_{HFS1} - \Delta E_{HFS2} = A(J) \hbar^2 F \Rightarrow \text{60MHz}       \\
     & \Delta E_{HFS2} - \Delta E_{HFS3} = A(J) \hbar^2 (F - 1) \Rightarrow \text{36MHz} \\
     & \Delta E_{HFS3} - \Delta E_{HFS4} = A(J) \hbar^2 (F - 2) \Rightarrow \text{17MHz} \\
\end{align*}

可以大致推测$F = 3$, 也即$I = \frac{3}{2}$

\section*{Answer for 3.12}

第一个PBS两个输出的正交偏振的光强为:
$$
    I_{1v} = \frac{I_0}{2} \text{, }\qquad I_{1h} = \frac{I_0}{2}
$$

第二个PBS两个输出的正交偏振的光强为:

\begin{align*}
     & I_{2v} = I_{1v} \cos^2 \theta = \frac{I_0}{2} \cos^2 \theta \\
     & I_{2h} = I_{1v} \sin^2 \theta = \frac{I_0}{2} \sin^2 \theta \\
\end{align*}

所以光子在第二个PBS垂直偏振光输出口的单光子计数器记录的概率为$\frac{1}{2}\cos^2 \theta$, 在第二个PBS水平偏振光输出口的单光子计数器记录的概率为$\frac{1}{2} \sin^2 \theta$

\section*{Answer for Lodon 4.2}

\begin{equation*}
    \begin{split}
        \left[\hat{a}, \left(\hat{a}^{\dagger}\right)^2\right] = & \left[\hat{a}, \hat{a}^\dagger\right] \hat{a}^\dagger + \hat{a}^\dagger \left[\hat{a}, \hat{a}^\dagger\right] \\
        =                                                        & 2 \hat{a}^\dagger                                                                                             \\
    \end{split}
\end{equation*}

\begin{equation*}
    \begin{split}
        \left[\left(\hat{a}\right)^2, \hat{a}^{\dagger}\right] = & \left[\hat{a}, \hat{a}^\dagger\right] \hat{a} + \hat{a}\left[\hat{a}, \hat{a}^\dagger\right] \\
        =                                                        & 2 \hat{a}
    \end{split}
\end{equation*}

\begin{equation*}
    \begin{split}
        \left[\hat{a}, \left(\hat{a}^{\dagger}\right)^n\right] = & \left[\hat{a}, \hat{a}^\dagger\right] \left(\hat{a}^\dagger\right)^{n-1} + \hat{a}^\dagger \left[\hat{a}, \left(\hat{a}^\dagger\right)^{n-1}\right]                             \\
        =                                                        & \left(\hat{a}^\dagger\right)^{n-1} + \hat{a}^\dagger \left[\hat{a}, \left(\hat{a}^\dagger\right)^{n-1}\right]                                                                   \\
        =                                                        & \left(\hat{a}^\dagger\right)^{n-1} + \hat{a}^\dagger \left[\left(\hat{a}^\dagger\right)^{n-2} + \hat{a}^\dagger \left[\hat{a}, \left(\hat{a}^\dagger\right)^{n-2}\right]\right] \\
        =                                                        & 2 \left(\hat{a}^\dagger\right)^{n-1} + \left(\hat{a}^\dagger\right)^2 \left[\hat{a}, \left(\hat{a}^\dagger\right)^{n-2}\right]                                                  \\
        =                                                        & \cdots                                                                                                                                                                          \\
        =                                                        & \left(n-2\right) \left(\hat{a}^\dagger\right)^{n-2} + \left(\hat{a}^\dagger\right)^{n-2} \left[\hat{a}, \left(\hat{a}^\dagger\right)^{2}\right]                                 \\
        =                                                        & \left(n-1\right) \left(\hat{a}^\dagger\right)^{n-1} + \left(\hat{a}^\dagger\right)^{n-1} \left[\hat{a}, \hat{a}^\dagger\right]                                                  \\
        =                                                        & n \left(\hat{a}^\dagger\right)^{n-1}
    \end{split}
\end{equation*}

\begin{equation*}
    \begin{split}
        \left[\left(\hat{a}\right)^n, \hat{a}^{\dagger}\right] = & \left[\hat{a}, \hat{a}^\dagger\right] \left(\hat{a}\right)^{n-1} + \hat{a} \left[\left(\hat{a}\right)^{n-1}, \hat{a}^\dagger\right]                             \\
        =                                                        & \left(\hat{a}\right)^{n-1} + \hat{a} \left[\left(\hat{a}\right)^{n-1}, \hat{a}^\dagger\right]                                                                   \\
        =                                                        & \left(\hat{a}^\dagger\right)^{n-1} + \hat{a}^\dagger \left[\left(\hat{a}\right)^{n-2} + \hat{a} \left[\left(\hat{a}\right)^{n-2}, \hat{a}^\dagger\right]\right] \\
        =                                                        & 2 \left(\hat{a}\right)^{n-1} + \left(\hat{a}\right)^2 \left[\left(\hat{a}\right)^{n-2}, \hat{a}^\dagger\right]                                                  \\
        =                                                        & \cdots                                                                                                                                                          \\
        =                                                        & \left(n-2\right) \left(\hat{a}\right)^{n-2} + \left(\hat{a}\right)^{n-2} \left[\left(\hat{a}\right)^{2}, \hat{a}^\dagger\right]                                 \\
        =                                                        & \left(n-1\right) \left(\hat{a}\right)^{n-1} + \left(\hat{a}\right)^{n-1} \left[\hat{a}, \hat{a}^\dagger\right]                                                  \\
        =                                                        & n \left(\hat{a}\right)^{n-1}
    \end{split}
\end{equation*}

\begin{equation*}
    \begin{split}
        \left[\hat{a}, \exp\left(\beta \hat{a}^\dagger\right)\right] = & \sum_n \left[\hat{a}, \frac{\beta^n}{n!} \hat{a}^\dagger\right]                        \\
        =                                                              & \sum_n \frac{\beta^n}{n!} n \left(\hat{a}^\dagger \right)^{n-1}                        \\
        =                                                              & \beta \sum_n \frac{\beta^{n-1}}{\left(n-1\right)!} \left(\hat{a}^\dagger \right)^{n-1} \\
        =                                                              & \beta \exp\left(\beta \hat{a}^\dagger\right)
    \end{split}
\end{equation*}

\section*{Answer for Lodon 4.3}

\begin{equation*}
    \hat{a}^\dagger \ket{n} = \sqrt{n+1}\ket{n+1} \Rightarrow \frac{\hat{a}^\dagger}{\sqrt{n+1}}\ket{n} = \ket{n+1}
\end{equation*}

\begin{equation*}
    \begin{split}
        \ket{n} = & \frac{\hat{a}^\dagger}{\sqrt{n}}\ket{n-1}                                                    \\
        =         & \frac{\left(\hat{a}^\dagger\right)^2}{\sqrt{n\left(n-1\right)}}\ket{n-2}                     \\
        =         & \cdots                                                                                       \\
        =         & \frac{\left(\hat{a}^\dagger\right)^{n-1}}{\sqrt{n\left(n-1\right) \cdots 3 \cdot  2}}\ket{1} \\
        =         & \frac{\left(\hat{a}^\dagger\right)^{n}}{\sqrt{n!}}\ket{1}                                    \\
        =         & \hat{N}(n) \ket{0}
    \end{split}
\end{equation*}


\section*{Answer for cmmutative of $\hat{X_1}$ and $\hat{X_2}$}



% \begin{equation*}
%     \left|\braket{\hat{A}\hat{B}}\right|^2 \leqslant  \braket{\hat{A}^2} \braket{\hat{B}^2}
% \end{equation*}

有施瓦茨不等式:
\begin{equation*}
    \begin{split}
        \braket{\hat{A}^2}\braket{\hat{B}^2} \geqslant & \left|\braket{AB}\right|^2                                                                                   \\
        =                                              & \left|\braket{\frac{AB + BA}{2} + \frac{AB - BA}{2}}\right|^2                                                \\
        =                                              & \left|\braket{\frac{AB + BA}{2}}\right|^2 + \left|\braket{\frac{AB - BA}{2}}\right|^2                        \\
        =                                              & \frac{1}{4}\left(\left|\bra{\left\{A, B\right\}}\right|^2 + \left|\braket{\left[A, B\right]}\right|^2\right) \\
        \geqslant                                      & \frac{1}{4} \left|\braket{\left[A, B\right]}\right|^2
    \end{split}
\end{equation*}

有:
\begin{equation*}
    \begin{split}
        \left[\Delta \hat{X_1}, \Delta \hat{X_2}\right] = & \left[\hat{X_1} - \braket{\hat{X_1}}, \hat{X_2} - \braket{\hat{X_2}}\right] \\
        =                                                 & \left[\hat{X_1}, \hat{X_2}\right]                                           \\
        =                                                 & \frac{i}{2}
    \end{split}
\end{equation*}

% \begin{equation*}
%     \begin{split}
%         \left(\Delta \hat{O}\right)^2 = & \braket{\left(\hat{O} - \braket{\hat{O}}\right)^2}                   \\
%         =                               & \braket{\hat{O}^2 + \braket{\hat{O}}^2 - 2 \braket{\hat{O}} \hat{O}} \\
%         =                               & \braket{\hat{O}^2}- \braket{\hat{O}}^2
%     \end{split}
% \end{equation*}

便可得到目标不确定关系:
\begin{equation*}
    \left(\Delta \hat{X_1}\right)^2 \cdot \left(\Delta \hat{X_2}\right)^2  \geqslant \frac{1}{16}  \Rightarrow \Delta \hat{X_1} \cdot \Delta \hat{X_2} \geqslant \frac{1}{4}
\end{equation*}


\end{document}
